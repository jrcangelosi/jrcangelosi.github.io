%%%%%%%%%%%%%%%%%%%%%%%%%%%%%%%%%%%%%%%%%
% "ModernCV" CV and Cover Letter
% LaTeX Template
% Version 1.11 (19/6/14)
%
% This template has been downloaded from:
% http://www.LaTeXTemplates.com
%
% Original author:
% Xavier Danaux (xdanaux@gmail.com)
%
% License:
% CC BY-NC-SA 3.0 (http://creativecommons.org/licenses/by-nc-sa/3.0/)
%
% Important note:
% This template requires the moderncv.cls and .sty files to be in the same 
% directory as this .tex file. These files provide the resume style and themes 
% used for structuring the document.
%
%%%%%%%%%%%%%%%%%%%%%%%%%%%%%%%%%%%%%%%%%

%----------------------------------------------------------------------------------------
%	PACKAGES AND OTHER DOCUMENT CONFIGURATIONS
%----------------------------------------------------------------------------------------

\documentclass[11pt,a4paper,sans]{moderncv} % Font sizes: 10, 11, or 12; paper sizes: a4paper, letterpaper, a5paper, legalpaper, executivepaper or landscape; font families: sans or roman

\moderncvstyle{classic} % CV theme - options include: 'casual' (default), 'classic', 'oldstyle' and 'banking'
\moderncvcolor{blue} % CV color - options include: 'blue' (default), 'orange', 'green', 'red', 'purple', 'grey' and 'black'

\usepackage{soul}
\usepackage[scale=0.8]{geometry} % Reduce document margins
\setlength{\hintscolumnwidth}{3.5cm} % Uncomment to change the width of the dates column
%\setlength{\makecvtitlenamewidth}{10cm} % For the 'classic' style, uncomment to adjust the width of the space allocated to your name

%----------------------------------------------------------------------------------------
%	NAME AND CONTACT INFORMATION SECTION
%----------------------------------------------------------------------------------------

\firstname{} % Your first name
\familyname{Jonathan Cangelosi} % Your last name

% All information in this block is optional, comment out any lines you don't need
%\title{Curriculum Vitae}
%\address{}{}
%\mobile{Contact: (225) 329-3718}
\email{jcange@stanford.edu}
\homepage{jrcangelosi.github.io}
%\homepage{www.pitt.edu/~shh69}{http://www.pitt.edu/$\sim$shh69} % The first argument is the url for the clickable link, the second argument is the url displayed in the template - this allows special characters to be displayed such as the tilde in this example
%\extrainfo{additional information}
% \photo[70pt][0.4pt]{pictures/daniele} % The first bracket is the picture height, the second is the thickness of the frame around the picture (0pt for no frame)
%\quote{"A witty and playful quotation" - John Smith}

%----------------------------------------------------------------------------------------

\begin{document}

\makecvtitle % Print the CV title



\vspace{-7mm}

\section{\textbf{Education}}

\cvitem{May 2023 -- May 2025}{\textbf{Rice University,} Houston, TX}
\cvitem{}{\textbf{Ph.D. in Computational and Applied Mathematics}}
\cvitem{}{Advisor: Matthias Heinkenschloss}
\cvitem{}{Thesis: \textit{An Adaptive Surrogate Model Refinement (ASMR) Framework for Simulation and Optimization of Dynamical Systems}}
\cvitem{}{Link: \url{https://hdl.handle.net/1911/118438}}
\cvitem{}{Defended April 4, 2025. Degree conferred May 10, 2025.}
\vspace{1em}

\cvitem{Aug 2020 -- May 2023}{\textbf{Rice University,} Houston, TX}
\cvitem{}{\textbf{M.A. in Computational and Applied Mathematics}}
\cvitem{}{Advisor: Matthias Heinkenschloss}
\cvitem{}{Thesis: \textit{Trajectory Optimization of Hypersonic Vehicles via a Radau Pseudospectral Method}}
\cvitem{}{Link: \url{https://hdl.handle.net/1911/114902}}
\cvitem{}{Defended December 5, 2022. Degree conferred May 6, 2023.}
\vspace{1em}


\cvitem{Aug 2016 -- Dec 2019}{\textbf{Louisiana State University,} Baton Rouge, LA}
\cvitem{}{\textbf{B.S. in Mathematics (with Honors),} December 2019.}
\cvitem{}{\textbf{B.S. in Computer Science (with Honors),} December 2019.}
\vspace{1em}







\vspace{-5mm}

\section{\textbf{Research Experience}}

\cvitem{Jul 2021 -- May 2025}{\textbf{Research Assistant,} Department of Computational Applied Mathematics and Operations Research (CMOR), Rice University}
\cvitem{}{\textbullet \; Implemented pseudospectral optimal control software in Python that uses the interior point solver IPOPT, leveraging sparsity of the constraint Jacobian and Lagrangian Hessian matrices for computational speedup and memory savings. Also used JAX for automatic differentiation and vectorization.}
\cvitem{}{\textbullet \; Derived rigorous sensitivity analysis results in a function space setting for simulation and optimization of dynamical systems with black-box functions. Work directly contributed two preprinted articles (one of which has been accepted by SIAM Numerical Analysis).}
\cvitem{}{\textbullet \; Developed an adaptive surrogate-assisted approach for simulation and optimization of dynamical systems that is sensitivity-driven, deterministic, parameter-free, and grounded in optimal control theory.}
\vspace{1em}

\cvitem{Dec 2021 -- Now}{\textbf{Multidisciplinary University Research Initiative (MURI)} }
\cvitem{}{Lead PI: Charbel Farhat, Stanford University}
\cvitem{}{Project title: \textit{A Robust Multi‐Physics Design Analysis and Optimization Framework for Hypersonic Systems Grounded in Rigorous Model Reduction}}
%\vspace{1em}
\cvitem{}{\textbullet \; Implemented a computational framework for trajectory optimization of one-body hypersonic vehicles using data-driven surrogate models for aerodynamic coefficients. Work directly contributed one conference paper to AIAA SciTech Forum 2024, co-authored by Jacob Needels and Dr.~Juan Jose Alonso, Stanford University.}
\vspace{1em}








\vspace{-5mm}

\section{\textbf{Publications and Preprints}}

\cvitem{Sep 2025}{\textbf{J. R. Cangelosi,} M. Heinkenschloss. \textit{Sensitivity-driven adaptive surrogate modeling for simulation and optimization of dynamical systems.} \url{https://arxiv.org/abs/2509.04651}.}
\vspace{1em}

\cvitem{Jun 2025}{\textbf{J. R. Cangelosi,} M. Heinkenschloss. \textit{Sensitivity of optimal control solutions and quantities of interest with respect to component functions.} \url{https://arxiv.org/abs/2506.10804}.}
\vspace{1em}

\cvitem{Nov 2024}{\textbf{J. R. Cangelosi,} M. Heinkenschloss. \textit{Sensitivity of ODE solutions and quantities of interest with respect to component functions in the dynamics.} \url{https://arxiv.org/abs/2411.09655}. Revised Jul 2025. Accepted for publication in SIAM J. Numer. Anal.}
\vspace{1em}

\cvitem{Jan 2024}{\textbf{J. R. Cangelosi,} M. Heinkenschloss, J. T. Needels, J. J. Alonso. \textit{Simultaneous design and trajectory optimization for boosted hypersonic glide vehicles.} AIAA SciTech 2024 Forum. \url{https://arc.aiaa.org/doi/10.2514/6.2024-0375}}
\vspace{1em}









\vspace{-5mm}

\section{\textbf{Presentations}}

\cvitem{Mar 2025}{``Sensitivity-Driven Surrogate Model Refinement for Optimization of Dynamical Systems with Expensive Black-Box Functions.'' Minisymposium presentation, SIAM CSE 2025, Fort Worth, TX.}
\vspace{1em}

\cvitem{Feb 2025}{``Sensitivity-Driven Surrogate Model Refinement for Trajectory Optimization with Expensive Black-Box Functions.'' Poster presentation, Energy HPC Conference 2025, Rice University.}
\vspace{1em}

\cvitem{Feb 2025}{``Sensitivity-Driven Surrogate Modeling for Trajectory Optimization.'' Minisymposium presentation, Scientific Machine Learning Workshop @ Energy HPC Conference 2025, Rice University.} 
\vspace{1em}

\cvitem{Oct 2024}{``Sensitivity-Driven Surrogate Model Refinement for Efficient Computation of Quantities of Interest in Dynamical Systems.'' Minisymposium presentation, SIAM-TXLA 2024, Baylor University.} 
\vspace{1em}

\cvitem{Sep 2024}{``An Adaptive Surrogate Model Refinement Framework for Simulation and Optimization of Dynamical Systems.'' Minisymposium presentation, MORe 2024, University of California, San Diego.} 
\vspace{1em}

\cvitem{Aug 2024}{``An Adaptive Surrogate Model Refinement Framework for Optimization of Dynamical Systems.'' Minisymposium presentation, MOPTA 2024, Lehigh University.} 
\vspace{1em}

\cvitem{Jul 2024}{``Surrogate Model Refinement for Simulation of Dynamical Systems.'' Poster presentation, NSF CompMath PI Meeting 2024, University of Washington.} 
\vspace{1em}

\cvitem{Jan 2024}{``Simultaneous Design and Trajectory Optimization for Boosted Hypersonic Glide Vehicles.'' Technical paper presentation, AIAA SciTech 2024, Orlando, FL.} 
\vspace{1em}

\cvitem{Nov 2023}{``Adaptive Gaussian Process Modeling for Trajectory Simulation with Model Inexactness.'' Minisymposium presentation, SIAM-TXLA 2023, University of Louisiana at Lafayette.} 
\vspace{1em}

\cvitem{Nov 2022}{``Trajectory Optimization of Hypersonic Vehicles via a Radau Pseudospectral Method.'' Poster presentation, SIAM-TXLA 2022, University of Houston.} 
\vspace{1em}








\vspace{-5mm}

\section{\textbf{Teaching}}

\cvitem{Spring 2023, \\ Spring 2024}{\textbf{Teaching Assistant,} CMOR, Rice University}
\cvitem{}{Held weekly recitation sessions and office hours for Differential Equations in Science and Engineering (CMOR 304), plus occasional guest lectures.}
\vspace{1em}

\cvitem{Spring 2024}{\textbf{Kernel Methods Reading Group Facilitator,} CMOR, Rice University}
\cvitem{}{Gave lectures and demonstrations on kernel methods from theoretical and practical perspectives for interested undergraduate and graduate students.}
\vspace{1em}

\cvitem{Aug 2017 -- Dec 2019}{\textbf{Lead Tutor,} Center for Academic Success, Louisiana State University}
\cvitem{}{Tutored students in calculus, differential equations, linear algebra, discrete mathematics, real analysis, optimization, intro-level programming, data structures, algorithm analysis, and intro-level physics.}
\vspace{1em}

\vspace{-5mm}







\section{\textbf{Service}}

\cvitem{Summer 2024, Summer 2025}{\textbf{Research Training Group (RTG) Summer Math Days Volunteer,} Rice University}
\cvitem{}{Gave a presentation to high school students discussing how mathematicians reason about infinity, which is foundational to college-level mathematics.}
\vspace{1em}

\cvitem{Summer 2024}{\textbf{RTG Summer Internship Volunteer,} Rice University}
\cvitem{}{Created Jupyter notebooks to teach high school students about discretizing and solving numerical optimization problems in Python using software such as SciPy and Pyomo.}
\vspace{1em}

\cvitem{Aug 2023 -- May 2024}{\textbf{Graduate seminar organizer,} Rice University}
\cvitem{}{Invited speakers and arranged weekly research talks for graduate students in the department.}
\vspace{1em}

\cvitem{Aug 2020 -- May 2023}{\textbf{Grader,} Rice University}
\cvitem{}{Differential Equations for Science and Engineering, Computational Science I, Numerical Analysis I and II.}
\vspace{1em}

\cvitem{Fall 2021}{\textbf{Math Nights volunteer,} Rice Association for Women in Mathematics (AWM) Chapter, Rice University}
\cvitem{}{Assisted undergraduate students in applied mathematics courses such as calculus, matrix analysis, and numerical methods for PDEs.}
\vspace{1em}









\vspace{-5mm}

\section{\textbf{Certificates}}

\cvitem{}{\textbf{Graduate Certificate in Teaching and Learning,} Center for Teaching Excellence, Rice University}
\cvitem{}{Accredited 2-year certificate program in the scholarship of teaching and learning.}

\vspace{1em}







\vspace{-5mm}



\section{\textbf{Technical Skills}}

\cvitem{}{\textbullet \; Programming languages: Python, C++}
\cvitem{}{\textbullet \; Scientific computing: NumPy, SciPy, JAX, OpenMP, MPI}
\vspace{1em}



%----------------------------------------------------------------------------------------
%	COVER LETTER
%----------------------------------------------------------------------------------------

% To remove the cover letter, comment out this entire block

%\clearpage
%
%\recipient{HR Department}{Corporation\\123 Pleasant Lane\\12345 City, State} % Letter recipient
%\date{\today} % Letter date
%\opening{Dear Sir or Madam,} % Opening greeting
%\closing{Sincerely yours,} % Closing phrase
%\enclosure[Attached]{curriculum vit\ae{}} % List of enclosed documents
%
%\makelettertitle % Print letter title
%
%%\lipsum[1-3] % Dummy text
%
%\makeletterclosing % Print letter signature

%----------------------------------------------------------------------------------------

\end{document}




